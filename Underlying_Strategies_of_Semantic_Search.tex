% Created 2020-03-06 vie 07:57
% Intended LaTeX compiler: pdflatex
\documentclass[11pt]{article}
\usepackage[utf8]{inputenc}
\usepackage[T1]{fontenc}
\usepackage{graphicx}
\usepackage{grffile}
\usepackage{longtable}
\usepackage{wrapfig}
\usepackage{rotating}
\usepackage[normalem]{ulem}
\usepackage{amsmath}
\usepackage{textcomp}
\usepackage{amssymb}
\usepackage{capt-of}
\usepackage{hyperref}
\usepackage[round]{natbib}
\usepackage{geometry}
\usepackage{helvet}
\geometry{left=3.5cm, right=2.5cm, top=2.5cm, bottom=2.5cm}
\linespread{1.25}
\author{nicoluarte}
\date{\today}
\title{}
\hypersetup{
 pdfauthor={nicoluarte},
 pdftitle={},
 pdfkeywords={},
 pdfsubject={},
 pdfcreator={Emacs 26.3 (Org mode 9.2.6)}, 
 pdflang={English}}
\begin{document}


\section{Agradecimientos}
\label{sec:org4af23e6}

Agradezco a CONICYT Proyecto Fondecyt Regular N°1170292 por la financiación de esta investigación.

\newpage
\section{Resumen}
\label{sec:orge11aae0}

Decisions are part of our daily lives, and it appears as if some kind of process
is evaluating second to second all of our options. In all of such situations a
big question arises, should I go for the well-known option or should I take my
chances a look for a new one?. This \textbf{exploration-exploitation dilemma} is also
present in both, foraging for resources and \textbf{semantic search}. As such, both
problems can be seen as decision-making processes where resources and semantic
contents locations are unknown, and somehow one must establish an efficient criterion
for searching in an efficient way. Certain search patterns, which are
ubiquitous across many taxa, seems to provide an optimal way for foraging
through a previously unknown environment. Given that both semantic search and
foraging share similarities, an evolutionary co-option of the mechanisms
controlling foraging for semantic search is discussed. Underlying strategies for
searching through patchy environments, neural implementations of
exploration-exploitation control and internal aspects of foraging are discussed
in hopes of providing an evolutionary framework for semantic search research.

\newpage
\section{Tabla de contenido}
\label{sec:org352f73b}
\setcounter{tocdepth}{2}
\tableofcontents
\newpage
\section{Índices de ilustraciones}
\label{sec:org2c35dc2}

\newpage
\section{Introduction}
\label{sec:org001eca2}
\subsection{Semantic search}
\label{sec:org3baebef}

Semantic memories are memories about the meaning of things, this conceptual
knowlegde allows us to interact and recognize objects, plan the future and
remember the past \citep{binderNeurobiologySemanticMemory2011}. Given such pivotal
role, the way that we 'navigate' through such memories will determine how we
interact with the world. The space in which the 'navigation' occurs has been
called semantic space, this corresponds to an abstraction where semantic
memories are placed in a multi-dimensional space, and the conection between them
are defined by some vector assigning the relationship between each one in every
dimension \citep{lundProducingHighdimensionalSemantic1996a}. In humans, however,
that way that semantic memories organize into such space is not clear
\citep{benedekHowSemanticMemory2017}.  

Free recall tasks, which prompt the participant to recall as many objects
pertaining to a certain category in a limited amount of time, have observed a
'patchy' distribution of such memories \citep{hillsOptimalForagingSemantic2009},
this patchy distribution refers to a signicantly faster retrieval time when the
participant is within a certain category (which is determined beforehand), more
than when they're not. The idea of a semantic space, with distances between
semantic contents, was first developed by supervised semantic network modeling based on lexical
co-ocurrence \citep{lundProducingHighdimensionalSemantic1996a}, which found
correlations between the distances calculated by the model and human retrieval
times, giving the possibility to model human memory in such a way.

The specific way this 'distance' exists in the brain is not known, however,
earlier lesion studies showed that specific neurological damage affect
specific semantic categories
\citep{hillisCATEGORYSPECIFICNAMINGCOMPREHENSION1991}, pointing that this
categories have some physical distance between them. Functional neuroimaging
data points in a similar direction, but is not clear wether the structure
represents actual semantic categories or some modality-specifc sub-divisions
\citep{caramazzaOrganizationConceptualKnowledge2003,binderWhereSemanticSystem2009}.
Network science studies have indicated that, within category elements, present
lower average shortest path length, higher clustering coefficient and lower
network modularity, implying that semantic contents are distributed in dense
clusters pertaining to each category, and that access to items within a category
have efficient access between them compared to another category
\citep{benedekHowSemanticMemory2017}.

Given a semantic space with a clustered-patchy distribution, the question of how
to efficiently retrieve such contents arises. Free recall tasks points that
,inter-retrieval intervals (IRI), when participants are producing items within a given
category are signicantly shorter that when they are not
\citep{abbottRandomWalksSemantic2015a}. IRI on this type of task have shown to be
following a Lévy probability density distribution with an exponent near 2 when
the retrievals are more efficient \citep{rhodesHumanMemoryRetrieval2007a}.

Patterns describing such inter-retrieval intervals have been compared to that of
food-foraging \citep{rhodesHumanMemoryRetrieval2007a}, this suggest a notion of
distance between memory contents that has been observed when participants are
asked to represent in a 2D space such memories
\citep{montezRoleSemanticClustering2015}.
\subsection{The exploration-exploitation dilemma in foraging and semantic search}
\label{sec:orga98a787}

Human semantic memory have been represented by relatively simple networks, where
semantic contents are de 'nodes' and the proximity metric (categorical or
associative similarity) is represented in the edges of this network. Such models
and empirical evidence in human recall tasks
\citep{hillsOptimalForagingSemantic2012a}, show that search within closely related
contents (or withing a category) are retrieved faster that when such items are
outside such category. However, the number of contents for each category are
finite, so certain strategy must be employed in order to guide the exploration
of this contents. Given this finite condition, a sequential decision making
problem is at face, at every step the decision to stay within a category or to
change to another (possibly more profitable category) must be made, and such
decision will impact further decisions. This could be stated as a non-stationary
environment, where the expected reward for each category is being constantly
updated to an unknown value, and in order to determine such value the agent must
explore it's options. However, in order to profit, at some point in time, it
also must exploit a category in order to maximize it's reward. This is known as
the exploration-exploitation dillema.

This dilemma observes the question of wether the agent should extract resources
(retrieve a semantic content) from the current patch in which the value is known
(i.e. exploitation) or wether it should explore other patches to gain
information about their value. One can assume that the organism should pick the
patch/category with the highest recent value, but this could lead to selecting a
sub-optimal patch or category if better options exists, or if its calculated
value is simply subject to many high degrees of noise
\citep{hernandez-lealExplorationStrategyNonstationary2016}.

To solve the exploration/exploitation dilemma several algorithms and notions,
such as expected and unexpected uncertainty have been proposed
\citep{mehlhornUnpackingExplorationExploitation2015a}, where a sensible threshold
is given to the agent to control expectations of rewards, such that exploitation
of a given patch is not stopped by a single low reward in face of multiple high
rewards. These algorithms try to solve the many problems that arise when trying
to act within this dilemma. Deciding is not easy because, while in a
non-depleting patch, the patch exploitation would always be the optimal
decision, the problem turns more complicated when exploitation depresses current
resources, and the value of all patches is not known. Which in turn, adds that
the organism must somehow determine an optimal policy for exploring and
exploiting. Early experiments on birds showed a 50:50 sample algorithm, where
the bird sampled equally two patches and for the rest of the time stayed in the
better one \citep{krebsTestOptimalSampling1978}. More recent work has established
the notion of simple heuristic-like rules such as energy reserves to decide how
intensively forage for food \citep{higginsonTrustYourGut2018}. Furthermore it has
been pointed that humans use different algorithms depending on the situation,
this has been shown using stationary sequential tasks
\citep{gershmanDeconstructingHumanAlgorithms2018}. These findings suggest that the
exploration-exploitation dilemma is likely to be solved by using the properties
of the environment the organism is in
\citep{fawcettEvolutionDecisionRules2014,hernandez-lealExplorationStrategyNonstationary2016}.

Up to this point semantic search has been introduced as a type of search through
a cognitive space, which is ruled by the exploration-exploitation dilemma.
Furthermore, the type of space in which this ocurrs was compared to that of
physical foraging \citep{hillsOptimalForagingSemantic2009} because of the patchy
distribution of 'resources' found in both cases.

\newpage
\section{Models for a semantic search heuristic}
\label{sec:org9dc2b1a}

The problem stated, asks for a way that is cognitively plausible, for conducting
semantic searches. Such a way could be provided by the use of certain primitive
heuristics. A heuristic is a way to handle information, that ignores part of it,
is relatively fast in getting to a decision, and gets a 'good-enough solution'
that is often the best in many scenarios where the optimal one is intractable
\citep{gigerenzerWhyHeuristicsWork2008a}. To provide a candidate heuristic is
necessary to submit computational models that allows to finely define such
heuristic and make testeable predictions \citep{gigerenzerWhyHeuristicsWork2008a}.
To define the underlying heuristics of semantic search two main models will be
presented one 'rule based' and other of stochastic nature. Secondly, an
heuristic most be supported by some specific structure capable of producing the
behavior predicted by the heuristic, for that neural implementations of such
strategies will be discussed. 

\subsection{Rule-based behavior: optimal foraging}
\label{sec:org1b2d586}

A prime exponent of rule-based strategies is the optimal foraging theory (OFT).
The OFT is one of the most well-known theories in the area of foraging behavior
since it provides a model that assumes both an optimal forager and a patchy
environment \citep{bartumeusOptimalSearchBehavior2009a}. Within OFT, the forager
is constantly updating the value (resource acquisition rate) of staying in a
patch versus leaving it for another one, since it is naturally assumed that as
permanence time increases, the resources of the current patch depresses. Given
that a forager uses energy to get to another patch, and also wants to optimize
the ratio between energy expenditure and caloric gain, it must establish a rule
for doing so in an efficient way. This rule is given by the marginal value
theorem (MVT). MVT dictates that an optimal forager should leave its current
patch when the acquisition rate drops below that of the habitat average
\citep{charnovOptimalForagingMarginal1976a}.

OFT provides a policy for decision making during food foraging. The first part
of such policy deals with the problem of how much time should the forager stay
in a given patch, so as not to miss the opportunity of potentially better
patches and, at the same time, effectively attain the resources of the current
one. The second part is to determine the optimal movement pattern or path the
forager should take when traveling from patch to patch. OFT largely avoided this
issue (i.e. how to travel from patch to patch) assuming previous knowledge of
habitat resource distribution. However, this probably is not the case for most
organisms.


Cognitive, perceptive or habitat dependent limitations make such previous
knowledge difficult \citep{bartumeusOptimalSearchBehavior2009a}. Thus, the way
that an organism adjusts its search behavior is of utmost importance. Random
walks have been extensively used to evolutionary describe such patterns
\citep{bartumeusOptimalSearchBehavior2009a,fioritiLevyForagingDynamic2015,humphriesForagingSuccessBiological2012,jagerHowSuperdiffusionGets2013}.
Evolutionary arguments for certain types of random walks such as the Levy walk
\citep{wosniackEvolutionaryOriginsLevy2017} assume that evolution optimizes the
most efficient search strategies. However, this view has been criticized
\citep{pierceEightReasonsWhy1987}, mainly because is impossible to prove that a
given behavior was selected for a specific function, and as such test for
optimality proves to be extremely difficult. Furthermore, it has been noted that
theoretical driven predictions do not necessarily match the data
\citep{pierceEightReasonsWhy1987}. However, in contrast to these criticisms the
point of optimal models for its heuristic value is proposed
\citep{parkerOptimalityTheoryEvolutionary1990}.

\subsection{Random walks}
\label{sec:orge042878}

A random walk movement model is, basically, a stochastic process that defines
two elements: (1) the step length, which is drawn from a certain probability
distribution and, (2) a turning angle or direction change which is also drawn
from a particular distribution. In the Levy-walks, turning angles are drawn from
a uniform distribution and the step lengths from a Levy-stable distribution
\citep{bartumeusLEVYPROCESSESANIMAL2007a}. Modeling search patterns using this
kind of random walk can be viewed as neglecting the fact of memory, especially
in organisms with higher cognitive abilities, mainly because there’s is no
parameter that captures the fact that, through memory, an organism can recall
previously visited placed or patches with high amount resources and so on.
However, the scale of the search deserves special attention. As an organism
might gain significant knowledge from a restricted area but not from all his
habitat, and even in the case of substantial habitat knowledge, the interaction
of other organisms with the environment may spoil prediction from that previous
knowledge, forcing the need to constantly gather information
\citep{patrickBoldnessPredictsIndividual2017}. Hence, having an efficient random
process for the search pattern might be useful even when higher cognitive
capabilities are provided \citep{bartumeusLEVYPROCESSESANIMAL2007a}. Aside from
considering cognitive capabilities, an important point is to determine whether
such modeled patterns do have an internal process that generates them,
otherwise, such pattern would not provide much information of the organism´s
behavior, and would appear more as a coincidence. For instance, it has been
discussed whether Levy-walks can appear as a result of resources properties,
that is, when the organism previously knows a resource location, but the value
of this resource obeys a power-law distribution, rather than from an underlying
Levy-process within the organism \citep{benhamouHOWMANYANIMALS2007}.

\newpage
\section{A case of co-option}
\label{sec:org47ad16e}
\subsubsection{Introduce the concept of co-option, emphasis on behavioral or search traits}
\label{sec:org9d51dc7}
\subsubsection{From where semantic search is co-opted from ? introduce foraging}
\label{sec:orge78dc8c}
\subsubsection{How this came to be}
\label{sec:org9a89bb2}
\newpage
\section{Neural implementations}
\label{sec:orgb71928e}
\subsubsection{Once a strategy/heuristic is identified, it is necesary to identify the structure underlying it}
\label{sec:org573194f}
\subsubsection{Base on exploration exploitation dilemma}
\label{sec:org578b7b1}
\newpage
\section{State dependent foraging}
\label{sec:org8b9c31a}
\newpage
\section{Conclusions}
\label{sec:org0fab3aa}
\newpage
\section{References}
\label{sec:orgc8af6e9}

\bibliographystyle{apa}
\bibliography{references}
\end{document}