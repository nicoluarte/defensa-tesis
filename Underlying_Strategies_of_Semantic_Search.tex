% Created 2020-01-22 mié 19:41
% Intended LaTeX compiler: pdflatex
\documentclass[11pt]{article}
\usepackage[utf8]{inputenc}
\usepackage[T1]{fontenc}
\usepackage{graphicx}
\usepackage{grffile}
\usepackage{longtable}
\usepackage{wrapfig}
\usepackage{rotating}
\usepackage[normalem]{ulem}
\usepackage{amsmath}
\usepackage{textcomp}
\usepackage{amssymb}
\usepackage{capt-of}
\usepackage{hyperref}
\usepackage[round]{natbib}
\usepackage{geometry}
\usepackage{helvet}
\geometry{left=3.5cm, right=2.5cm, top=2.5cm, bottom=2.5cm}
\linespread{1.25}
\author{nicoluarte}
\date{\today}
\title{}
\hypersetup{
 pdfauthor={nicoluarte},
 pdftitle={},
 pdfkeywords={},
 pdfsubject={},
 pdfcreator={Emacs 26.3 (Org mode 9.2.6)}, 
 pdflang={English}}
\begin{document}


\section{Agradecimientos}
\label{sec:orgab4e33d}

Agradezco a CONICYT Proyecto Fondecyt Regular N°1170292 por la financiación de esta investigación.

\newpage
\section{Resumen}
\label{sec:org9582f6a}

Decisions are part of our daily lives, and it appears as if some kind of process
is evaluating second to second all of our options. In all of such situations a
big question arises, should I go for the well-known option or should I take my
chances a look for a new one?. This \textbf{exploration-exploitation dilemma} is also
present in both, foraging for resources and \textbf{semantic search}. As such, both
problems can be seen as decision-making processes where resources and semantic
contents locations are unknown, and somehow one must establish an efficient criterion
for searching in an efficient way. Certain search patterns, which are
ubiquitous across many taxa, seems to provide an optimal way for foraging
through a previously unknown environment. Given that both semantic search and
foraging share similarities, an evolutionary co-option of the mechanisms
controlling foraging for semantic search is discussed. Underlying strategies for
searching through patchy environments, neural implementations of
exploration-exploitation control and internal aspects of foraging are discussed
in hopes of providing an evolutionary framework for semantic search research.

\newpage
\section{Tabla de contenido}
\label{sec:org0993aa0}
\setcounter{tocdepth}{2}
\tableofcontents
\newpage
\section{Índices de ilustraciones}
\label{sec:org9e44a65}

\newpage
\section{Introduction}
\label{sec:org73b43ae}
\subsection{Semantic search}
\label{sec:orgad61c1d}
\subsubsection{What it is}
\label{sec:org304827d}
\subsubsection{Empirical evidence}
\label{sec:orga96d8d6}
\subsubsection{Observed patterns}
\label{sec:org6cc66f8}
\subsubsection{Justfication of 'space' or 'effort'}
\label{sec:org3292c49}
Semantic memories are memories about the meaning of things, this conceptual
knowlegde allows us to interact and recognize objects, plan the future and
remember the past \citep{binderNeurobiologySemanticMemory2011}. Given such pivotal
role, the way that we 'navigate' through such memories will determine the way we
interact with the world. The space in which the 'navigation' occurs has been
called semantic space, which corresponds to an abstraction where semantic
memories are placed in a multi-dimensional space and the conection between them
are defined by some vector assigning the relationship between each one in every
dimension \citep{lundProducingHighdimensionalSemantic1996a}. In humans, however,
that way that semantic memories organize into such space is not clear
\citep{benedekHowSemanticMemory2017}.  

Free recall tasks, which prompt the participant to recall as many objects
pertaining to a certain category in a limited amount of time, have observed a
'patchy' distribution of such memories \citep{hillsOptimalForagingSemantic2009},
this patchy distribution refers to a signicantly faster retrieval time when the
participant is within a certain category (which is determined beforehand). The
pattern describing such inter-response intervals have been compared to that of
food-foraging \citep{rhodesHumanMemoryRetrieval2007a}, this suggest a notion of
distance between memory contents that has been observed when participants are
asked to represent in a 2D space such memories \citep{montezRoleSemanticClustering2015} 








\subsection{Sequential decision making}
\label{sec:org7185ddf}
\subsubsection{What is the 'problem' present in semantic search (specific to retrieval tasks)}
\label{sec:org7029d48}
\subsubsection{Brief intro to sequential decision making an its issue (da paso a exploration-exploitation)}
\label{sec:orgdc55109}
\subsection{The exploration-exploitation dilemma in foraging and semantic search}
\label{sec:org626560d}
\subsubsection{Presentar el dilemma}
\label{sec:org3eb5538}
\subsubsection{Connect both through evidence}
\label{sec:org0b316dd}
\subsubsection{Connect both through logic}
\label{sec:org8498970}
\newpage
\section{Models for a Heuristic}
\label{sec:org8044340}
\subsubsection{Define heuristics clearly}
\label{sec:org8b1ed37}
\subsubsection{Put the question of what is the underlying heuristic}
\label{sec:org3196b27}
\subsubsection{Argue how a model could represent a heuristic}
\label{sec:org8db3261}
\subsubsection{Rule-based}
\label{sec:org6e1b815}
\subsubsection{Random walks}
\label{sec:orge2fd6c0}
\newpage
\section{A case of co-option}
\label{sec:orgb0b5b54}
\subsubsection{Introduce the concept of co-option, emphasis on behavioral or search traits}
\label{sec:org0cb1839}
\subsubsection{From where semantic search is co-opted from ? introduce foraging}
\label{sec:orgcf17438}
\subsubsection{How this came to be}
\label{sec:orga8392a9}
\newpage
\section{Neural implementations}
\label{sec:org57a64f5}
\subsubsection{Once a strategy/heuristic is identified, it is necesary to identify the structure underlying it}
\label{sec:org9bad9a7}
\subsubsection{Base on exploration exploitation dilemma}
\label{sec:org0af76b7}
\newpage
\section{State dependent foraging}
\label{sec:org26fa7c9}
\newpage
\section{Conclusions}
\label{sec:orgbe6015a}
\newpage
\section{References}
\label{sec:org77a5bd1}

\bibliographystyle{apa}
\bibliography{references}
\end{document}