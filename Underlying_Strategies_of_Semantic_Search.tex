% Created 2020-01-21 mar 19:23
% Intended LaTeX compiler: pdflatex
\documentclass[11pt]{article}
\usepackage[utf8]{inputenc}
\usepackage[T1]{fontenc}
\usepackage{graphicx}
\usepackage{grffile}
\usepackage{longtable}
\usepackage{wrapfig}
\usepackage{rotating}
\usepackage[normalem]{ulem}
\usepackage{amsmath}
\usepackage{textcomp}
\usepackage{amssymb}
\usepackage{capt-of}
\usepackage{hyperref}
\usepackage{geometry}
\usepackage{helvet}
\geometry{left=3.5cm, right=2.5cm, top=2.5cm, bottom=2.5cm}
\linespread{1.25}
\author{nicoluarte}
\date{\today}
\title{}
\hypersetup{
 pdfauthor={nicoluarte},
 pdftitle={},
 pdfkeywords={},
 pdfsubject={},
 pdfcreator={Emacs 26.3 (Org mode 9.2.6)}, 
 pdflang={English}}
\begin{document}

\tableofcontents


\section{Resumen}
\label{sec:orgf08216b}

Decisions are part of our daily lives, and it appears as if some kind of process
is evaluating second to second all of our options. In all of such situations a
big question arises, should I go for the well-known option or should I take my
chances a look for a new one?. This \textbf{exploration-exploitation dilemma} is also
present in both, foraging for resources and \textbf{semantic search}. As such, both
problems can be seen as decision-making processes where resources and semantic
contents locations are unknown, and somehow one must establish an efficient criterion
for searching in an efficient way. Certain search patterns, which are
ubiquitous across many taxa, seems to provide an optimal way for foraging
through a previously unknown environment. Given that both semantic search and
foraging share similarities, an evolutionary co-option of the mechanisms
controlling foraging for semantic search is discussed. Underlying strategies for
searching through patchy environments, neural implementations of
exploration-exploitation control and internal aspects of foraging are discussed
in hopes of providing an evolutionary framework for semantic search research.

\section{Tabla de contenido}
\label{sec:org9a1d744}

\section{Índices de ilustraciones}
\label{sec:org0b24782}
\end{document}